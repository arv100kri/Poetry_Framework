\documentclass[10pt, letter,twocolumn]{IEEEtran}

\usepackage{graphicx}
\newcommand{\bigO}{\ensuremath{\mathcal{O}}}
\usepackage[center]{caption}
\newcommand{\doctitle}{%
Computational Creativity/Modeling and Design}
\usepackage{textcomp}
\usepackage{listings}
\usepackage{comment}
\usepackage{fancyvrb}
\usepackage{booktabs}
\usepackage[usenames,dvipsnames]{color}
\usepackage{hyperref}
\usepackage{algorithm}
\usepackage{algpseudocode}
\hypersetup{
  colorlinks,
  citecolor=Violet,
  linkcolor=Black,
  urlcolor=Blue}
  
\begin{document}

\title{Poetry Framework: Recognize Generate and \textsc{Understand} Poetry}
	\author{\IEEEauthorblockN{Arvind Krishnaa Jagannathan, Greg Cobb,
	Michelle Scott, Sonal Danak} 
	\IEEEauthorblockA {
	\\[0.5cm] \textbf{\doctitle} \\
	\textsc{\textbf{Deliverable \#2:}}
	}
	}
\maketitle
\thispagestyle{empty}
\section{Problem}
Our problem statement was pretty well-formed right before the first deliverable. Our lofty goal is to be able to architect a framework which has the basic structures of poetry encoded in it. By that we mean that our system will be able to discern poetry from ordinary prose, recognize the different syntactic structures which determine the type of a poem and use this knowledge to categorize different poems which are given to the system as input. By having an ``understanding'' of the processes involved in generating poetry, our framework will also have a module/sub-system which can generate poetry. As we had mentioned earlier, the generation sub-system could utilize different corpora of text data, such as Wikipedia articles, a user's Twitter feed and so on to obtain rhyming pairs of phrases to
produce a rudimentary poem.

Although the core of our project's problem statement has not evolved, in the duration since the last deliverable, we have been able to redefine the scope of each of these sub-components of the framework, including the component which is responsible for assisting a novice poet in coming up with a ``first draft'' poem.

\subsection*{Core Framework}

\subsection*{Poetry Rule Engine}

\subsection*{Poetry Recognizer}

\subsection*{Poetry Generator}

\subsection*{Poetry Assistant}

\section{Architecture}
\section{Data Structures}
\section{Algorithms}
\section{Interface Diagram}

\bibliographystyle{unsrt}
\bibliography{myrefs}
\end{document}
\end{document}